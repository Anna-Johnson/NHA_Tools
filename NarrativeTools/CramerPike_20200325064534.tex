%This knitr document is called by the knit2pdf ....
\documentclass{article}\usepackage[]{graphicx}\usepackage[]{color}
% maxwidth is the original width if it is less than linewidth
% otherwise use linewidth (to make sure the graphics do not exceed the margin)
\makeatletter
\def\maxwidth{ %
  \ifdim\Gin@nat@width>\linewidth
    \linewidth
  \else
    \Gin@nat@width
  \fi
}
\makeatother

\definecolor{fgcolor}{rgb}{0.345, 0.345, 0.345}
\newcommand{\hlnum}[1]{\textcolor[rgb]{0.686,0.059,0.569}{#1}}%
\newcommand{\hlstr}[1]{\textcolor[rgb]{0.192,0.494,0.8}{#1}}%
\newcommand{\hlcom}[1]{\textcolor[rgb]{0.678,0.584,0.686}{\textit{#1}}}%
\newcommand{\hlopt}[1]{\textcolor[rgb]{0,0,0}{#1}}%
\newcommand{\hlstd}[1]{\textcolor[rgb]{0.345,0.345,0.345}{#1}}%
\newcommand{\hlkwa}[1]{\textcolor[rgb]{0.161,0.373,0.58}{\textbf{#1}}}%
\newcommand{\hlkwb}[1]{\textcolor[rgb]{0.69,0.353,0.396}{#1}}%
\newcommand{\hlkwc}[1]{\textcolor[rgb]{0.333,0.667,0.333}{#1}}%
\newcommand{\hlkwd}[1]{\textcolor[rgb]{0.737,0.353,0.396}{\textbf{#1}}}%
\let\hlipl\hlkwb

\usepackage{framed}
\makeatletter
\newenvironment{kframe}{%
 \def\at@end@of@kframe{}%
 \ifinner\ifhmode%
  \def\at@end@of@kframe{\end{minipage}}%
  \begin{minipage}{\columnwidth}%
 \fi\fi%
 \def\FrameCommand##1{\hskip\@totalleftmargin \hskip-\fboxsep
 \colorbox{shadecolor}{##1}\hskip-\fboxsep
     % There is no \\@totalrightmargin, so:
     \hskip-\linewidth \hskip-\@totalleftmargin \hskip\columnwidth}%
 \MakeFramed {\advance\hsize-\width
   \@totalleftmargin\z@ \linewidth\hsize
   \@setminipage}}%
 {\par\unskip\endMakeFramed%
 \at@end@of@kframe}
\makeatother

\definecolor{shadecolor}{rgb}{.97, .97, .97}
\definecolor{messagecolor}{rgb}{0, 0, 0}
\definecolor{warningcolor}{rgb}{1, 0, 1}
\definecolor{errorcolor}{rgb}{1, 0, 0}
\newenvironment{knitrout}{}{} % an empty environment to be redefined in TeX

\usepackage{alltt}
\usepackage[utf8]{inputenc} %\UseRawInputEncoding
\usepackage{fontspec}
\setmainfont{Gill Sans MT}
\pdfmapfile{=pdftex35.map} %I think this fixes some MikTex font reading issues
\usepackage[margin=10pt,font=small]{caption}
\usepackage{geometry}
\usepackage{xcolor}
\usepackage{longtable,booktabs,threeparttablex}
\usepackage{wrapfig}
\usepackage{caption}
\usepackage{subcaption}
\usepackage{url}
\urlstyle{same}
\usepackage{graphicx}
\graphicspath{ {../../z_BaseImages/}{../../z_BaseImages/EO_images/}{photos/} }
\usepackage[style=authoryear,hyperref=false]{biblatex}
\addbibresource{../../citations/PNHP_refs.bib}
%\addbibresource{pnhp_refs.bib}
%\addbibresource{P:/Conservation Programs/Natural Heritage Program/ConservationPlanning/NaturalHeritageAreas/_NHA/citationspnhp_refs}
\usepackage{enumitem}
\setlist{nolistsep}
\usepackage{fancyhdr} %for headers,footers
% \usepackage{float}
\usepackage{hyperref}
\hypersetup{
    colorlinks=true,
    linkcolor=blue,
    filecolor=magenta,      
    urlcolor=blue,
}
\usepackage{lastpage}

\geometry{letterpaper, top=0.45in, bottom=0.75in, left=0.75in, right=0.75in}
\pagestyle{fancy} \fancyhf{} \renewcommand\headrulewidth{0pt} %strip default header/footer stuff

\setlength\intextsep{0pt}

%add footers
\lfoot{
 \small   %small font. The double slashes is newline in fancyhdr
 \textcolor{gray}{Cramer Pike Natural Heritage Area\\Pennsylvania Natural Heritage Program }
}
\rfoot{
 \small  
 \textcolor{gray}{page \thepage \ of \ \pageref*{LastPage}}
}
\IfFileExists{upquote.sty}{\usepackage{upquote}}{}
\begin{document}
%\raggedright
\catcode`\_=11

% Header
\noindent
\textbf{\LARGE{Cramer Pike NHA}}\\
\large A site of \underline{State Significance} \
\medskip \\

% % image
NULL


% \begin{wrapfigure}{R}{0.5\textwidth} % [13]
%   \includegraphics[width=0.5\textwidth]{nha_photos$P1F} %  
%   \captionsetup{labelformat=empty, justification=raggedright}
%   <<label=captionfunction, echo=FALSE, results='asis'>>=
%   capfun <- as.character(paste(nha_photos$P1C, "\\textcolor{gray}{","Photo: ", nha_photos$P1N,"}\\\\", sep=" "))
%   @
%   \caption{capfun}
% \end{wrapfigure}


% Site Description
\normalsize
\noindent
This site supports a sensitive species of concern. The species occurs in a small wetland in a right-of-way, where the open canopy may help it persist.  The species might also be present in other wetlands and seeps along the stream valley. \\ \par\noindent The wetland includes a variety of species, including skunk cabbage (\textit{\textit{Symplocarpus} foetidus}), jewelweed (\textit{Impatiens} sp.), cinnamon fern (\textit{\textit{Osmunda} cinnamomea}), sensitive fern (\textit{\textit{Onoclea} sensibilis}), swamp aster (\textit{\textit{Symphyotrichum} puniceum}), fringed sedge (\textit{\textit{Carex} crinita}), lurid sedge (\textit{\textit{Carex} lurida}), awl-fruited sedge (\textit{\textit{Carex} stipata}), northern long sedge (\textit{\textit{Carex} folliculata}), common rush (\textit{\textit{Juncus} effusus}), rice cut-grass (Leerzia oryzoides), and fowl mannagrass (\textit{\textit{Glyceria} striata}).\\\\
% paragraph about significance ranks
This site has been rated as  significant due 
% calculates what text to say

The species tracked by PNHP present at this NHA include:
\medskip
% Species Table
\begin{ThreePartTable}
\renewcommand\TPTminimum{\textwidth}
%% Arrange for "longtable" to take up full width of text block
\setlength\LTleft{0pt}
\setlength\LTright{0pt}
\setlength\tabcolsep{0pt}

\begin{TableNotes}
    \item [1] See the PNHP website (\href{http://www.naturalheritage.state.pa.us/rank.aspx}{http://www.naturalheritage.state.pa.us/rank.aspx}) for an explanation of PNHP ranks and legal status. A legal status in parentheses is a status change recommended by the Pennsylvania Biological Survey.
    \item [2] See NatureServe website (\href{http://www.natureserve.org/explorer/eorankguide.htm}{http://www.natureserve.org/explorer/eorankguide.htm}) for an explanation of quality ranks.

    \item [3] This species is not named by request of the jurisdictional agency responsible for its protection.\end{tablenotes}
\end{TableNotes}

\begin{longtable}{ l @{\extracolsep{\fill}} *{6}{c} }
\toprule
\textbf{Species or Natural Community Name} &  & \textbf{Global}\tnote{1} & \textbf{State}\tnote{1} & \textbf{PA Legal Status} & \textbf{Last Observed}	& \textbf{Quality}\tnote{2} \\
%Country & N obs  & Total years & No degree & High school & Some college,+ & smth\tnote{a} & smth\tnote{b} \\
%& & of education & & & & prestige score & income score \\ 
\midrule
\endhead

\midrule[\heavyrulewidth]
\multicolumn{7}{r}{\textit{table continued on next page}}\\
\endfoot  

\midrule[\heavyrulewidth]
\insertTableNotes  % tell LaTeX where to insert the table-related notes
\endlastfoot

Sensitive Species of Concern A\tnote{3} &\includegraphics[width=0.15in]{Sensitive.png}&--&--&-- (--)&2012&E\\


\end{longtable}   
\end{ThreePartTable}


%%% Threats and Species Recommendations %%%
\medskip
\noindent
\textbf{\underline{Threats and Species Recommendations}}\\\\
\normalsize 
\noindent Disturbances should be minimized at this site. The surrounding forest should be left intact. Invasive species are a potential concern at this site, especially since a new ROW was recently cleared 70 meters from the species of concern. Specific threats and stresses to the elements present at this site, as well as conservation actions, include:  
\begin{itemize}
\item Conversion of the surrounding forest to other land uses is a threat. Protect the remaining habitat from conversion to housing, pipelines, and other land uses.\item Invasive exotic plant species crowd out native plants if left unchecked. Japanese stilt grass (\textit{Microstegium} vinineum) is present at this site, although as an upland plant it will not directly affect this wetland. Invasive wetland plants should be watched for, and eradicated if found.\item The wetland is probably fed by groundwater. By avoiding large withdrawals of groundwater, groundwater levels can be maintained. Groundwater quality can be compromised by septic systems and agricultural fertilizers and pesticides.
\end{itemize}

\bigskip

%%% Location and Additional Infomation %%% 
\pagebreak[1]
\noindent\textbf{\underline{Location}}\\\\
\smallskip
\textbf{Municipalities:} Indiana County: East Wheatfield Township \\
\textbf{USGS quads:} Vintondale \\
\textbf{Previous CNHI reference:} NA \\
\textbf{Associated NHAs:} None \\ 
\textbf{Overlapping Protected Lands:} This site is not documented as overlapping with any Federal, state, or locally protected land or conservation easements. \\
\textbf{Acreage:} 0.8 acres \\  
  
%%% References %%% 
\noindent\textbf{\underline{References}}
%\bibliographystyle{plain}
\printbibliography[heading=none]



\bigskip
% the next line moves the recommended citation to the bottom of the page
\vspace*{\fill}
%%%  Recommended Citation %%% 
\setlength{\fboxsep}{5pt}
\fbox{
\begin{minipage}[c]{0.2\linewidth}
\includegraphics[width=1.0\linewidth]{PNHP_New_Logo_cmpact}%png logo file at repository root
\end{minipage}%
\begin{minipage}[c]{0.75\linewidth}

\textit{Please cite this Natural Heritage Area as:} \\
Pennsylvania Natural Heritage Program. 2020. Cramer Pike NHA. . Created on 25 Mar 2020. . Available at: \href{http://www.naturalheritage.state.pa.us/inventories.aspx}{http://www.naturalheritage.state.pa.us/inventories.aspx}   

\end{minipage}
}

\newpage
insert map here

\end{document}
